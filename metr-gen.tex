\begin{definition*}
  \emph{Метрикой} называется неотрицательная вещественнозначная функция двух
  аргументов $ \rho(x, y)\colon A \to \mathbb R$,
  определённая на произвольном множестве $ A $, со свойствами
  \begin{enumerate}[label=\alph*)]
    \item $ \rho(x, y) = 0 \Leftrightarrow x = y $,
    \item $ \rho(x, y) = \rho(y,x) $ (\textsc{симметричность}),
    \item $ \rho(x, z) \leqslant \rho(x, y) + \rho(y,z) $ (\textsc{неравенство
      треугольника})
  \end{enumerate}
  для любых $ x,y,z\in A $.

  Множество с введённой метрикой называют \emph{метрическим пространством}.
\end{definition*}

