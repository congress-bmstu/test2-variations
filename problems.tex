\begin{problem}
Теоретический вопрос.
\end{problem}

\begin{problem}
 Найти экстремали задачи с подвижными концами.  
 \[
   \int\limits_{0}^{1} ((y')^2 + 2yy' - 32y^2)\,dx \to \mathrm{extr}.
 \]
 
\begin{solution}
  Запишем уравнение Эйлера:  
  \[
    F_y - \frac{d}{dx} F_{y'} = 0.
  \]
 Получим  
 \[
     2y' - 64y - \frac{d}{dx}(2y' + 2y) = - 64y - 2y'' = 0,
 \]
 или $ y'' + 32y = 0 $. Решим это уравнение.  
 \[
   \lambda^2 + 32 = 0 \implies \lambda = \pm\sqrt{-32} = \pm 4i\sqrt{2},\\
 \]
 Поскольку корни чисто мнимые, общее решение имеет вид  
 \begin{align*}
   y(x) &= C_1 \cos 4\sqrt{2}x + C_2 \sin 4\sqrt{2}x,\\
   y'(x) &= 4\sqrt2(-C_1\sin4\sqrt2 x + C_2\cos4\sqrt2 x).
 \end{align*}
 Запишем условия трансверсальности: $ F_{y'} \big|_{x=a} = 0 $, $ F_{y'}\big|_{x=b}=0 $,
 где $ a = 0 $, $ b = 1 $.

 В нашем случае
 \begin{multline*}
   F_{y'} = 8\sqrt{2}(-C_1\sin4\sqrt2 x + C_2 \cos4\sqrt2 x) + 2C_1\cos 4\sqrt2
   x + 2C_2 \sin4\sqrt2 x = \\ =
   (-8\sqrt2\sin4\sqrt2 x + 2\cos4\sqrt2 x)C_1 + (8\sqrt2 \cos 4\sqrt2 x +
   2\sin4\sqrt2 x) C_2.
 \end{multline*}
Получаем систему
\[
    \begin{cases}
      2C_1 + 8\sqrt2 C_2 = 0, \\
   (-8\sqrt2\sin4\sqrt2  + 2\cos4\sqrt2 )C_1 + (8\sqrt2 \cos 4\sqrt2  +
   2\sin4\sqrt2 ) C_2 = 0.
    \end{cases}
\]
Видно, что определитель $ \det A = 132 \sin4 \sqrt{2} \approx
  -77.375 \neq 0 $, поэтому система имеет единственное, нулевое решение. Вместе
  с решением нулевой будет экстремаль задачи.
\end{solution}
\end{problem}

\begin{problem}
  Найти экстремали изопериметрической задачи.  
  \[
    \int\limits_{0}^{2} (y')^2\,dx \to \mathrm{extr}, \quad
    \int\limits_{0}^{2} xy\,dx = 1, \quad
    y(0) = 0, \quad
    y(2) = 0.
  \]
  
\begin{solution}
  Подынтегральную функцию первого интеграла обозначим $ F $, второго --- $ G $.
  Найдём функцию $ R = F + \lambda G $ и составим для неё уравнение Эйлера
  \begin{gather*}
      R= (y')^2 + \lambda xy,\\
      R_y - \frac{d}{dx} R_{y'} = 0,\\
      \lambda x - 2y'' = 0.
  \end{gather*}
Получим ответ $ y = \frac{\lambda}{12} x^3 + C_1 x + C_2 $. Причём, исходя из
граничных условий, $ C_2 = 0 $, $ C_1 = - \lambda/3 $, и окончательно,  
\[
    y = \frac{\lambda}{12}x^3 - \frac{\lambda}{3} x.
\]
Подставим полученную функцию во второй интеграл.  
\[
  \lambda\int\limits_{0}^{2}\frac{x^4}{12} - \frac{x^2}{3}\,dx = 1,
\]
откуда $ \lambda = -45/16 $. Итоговый ответ: 
\[
    \boxed{
      y(x) = - \frac{15 x^{3}}{64} + \frac{15 x}{16}.
    }
\]

\end{solution}
\end{problem}

\begin{problem}
  Найти экстремали задачи Больца. 
  \[
    \int\limits_{0}^{1} (y')^2\,dx + y(0)^2 - y(0) + y(1)^2 \to \mathrm{extr}.
  \]
\begin{solution}
  Обозначим подынтегральную функцию как $ F $, а остаток как $ \psi(y(0), y(1))
  $. Составляем и решаем уравнения Эйлера для $ F $: 
  \[
      -2y'' = 0 \implies y = C_1x+ C_2, \quad y' = C_1.
  \]
  Запишем условия трансверсальности: $ F_{y'} \big|_{x=a} = \psi_{y(a)} $, $
  F_{y'}\big|_{x=b}= -\psi_{y(b)} $,
 где $ a = 0 $, $ b = 1 $. Согласно им $ 2y'(0) = 2y(0) - 1 $, а $ 2y'(1) =
 -2y(1) $. Тогда  
 \[
     \begin{cases}
       2C_1 = 2C_2 - 1, \\
       2C_1 = -2C_1 - 2C_2,
     \end{cases} \implies
     \begin{cases}
       2C_1 - 2C_2 = -1, \\
       4C_1 + 2C_2 = 0.
     \end{cases}
 \]
Методом Крамера получаем $ C_1 = -1/6 $, $ C_2 = 1/3 $. Экстремаль равна 
\[
    \boxed{
      y(x) = - \frac{1}{6} x + \frac{1}{3}.
    }
\]
\end{solution}
\end{problem}

\begin{problem}
  Исследовать задачу на наличие слабого минимума или слабого максимума. 
  \[
    \int\limits_{0}^{1} yy' - (y')^2\,dx \to \mathrm{extr}, \quad y(0) = 0,
    \quad y(1) = 1.
  \]
\begin{solution}
Необходимым условием слабого минимума (максимума) является удовлетворение
функцией
уравнению Эйлера.


  Запишем уравнение Эйлера для подынтегральной функции $ F $.  
  \[
    2y'' = 0, \quad \text{откуда } y = C_1 x + C_2.
  \]
 Учитывая граничные условия, $ C_2 = 0 $, $ C_1 = 1 $. Значит, $ y = x $. 
 % (Ниже
 % выяснится, что решение уравнения Эйлера можно было и не искать.)

Достаточным условием слабого минимума (максимума) для допустимой экстремали является 
\begin{enumerate}
  \item выполнение усиленного условия Лежандра $ F_{y'y'} > 0 $ ($ F_{y'y'} < 0
    $),
  \item выполнение усиленного условия Якоби, то есть
    \begin{enumerate}
      \item существование решения уравнения Якоби,
      \item необратимость в нуль решения уравнения Якоби на полуинтервале $ (a,
        b]$.
    \end{enumerate}
\end{enumerate}


 Проверим усиленное условие Лежандра. Рассмотрим функцию $ F_{y'y'} = -2 < 0 $. Поскольку она всегда строго
 меньше нуля, то возможен слабый максимум. Запишем уравнения Якоби 
 \[
   \begin{cases}
     \left( F_{yy} - \frac{d}{dx} \left( F_{yy'} \right)  \right) u - \frac{d}{dx}
     \left( F_{y'y'} u' \right) = 0,\\
     u(a) = 0,\\
     u'(a) = 1,
   \end{cases}
 \]
 где $ a = 0 $ в нашем случае. В нашем случае также $ F_{yy} = 0 $, $
 \frac{d}{dx} F_{yy'} = 0 $ и $ \frac{d}{dx}F_{y'y'} = -2u'' $. Тогда  
 \[
     \begin{cases}
       2u'' =0, \\
       u(0) = 0,\\
       u'(0) = 1.
     \end{cases}
 \]
Отсюда снова $ u = x $. Для того чтобы функция $ y(x) $ была слабым максимумом,
теперь необходимо, чтобы $ u(x) $ не обращалась в нуль на полуинтервале $ (a, b] =
(0, 1] $. Наша функция удовлетворяет этому условию, поэтому \fbox{$ y(x) = x $
является слабым максимумом.} 

% Если бы $ u(x) $ обращалась в нуль на указанном полуинтервале в точке $ x_0 $,
% то эту точку мы бы назвали \emph{сопряжённой к $ a $ точкой}.
% Поскольку решение уравнения Якоби существует, то
% функция $ y $ действительно может являеться слабым максимумом. 
\end{solution}
\end{problem}



