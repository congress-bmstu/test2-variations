\vspace{-0.7em}
\que{Функционал. Класс допустимых функций. Функционал в простейшей задаче
вариационного исчисления. Операции сложения функций и умножения функции на число
в классе допустимых функций. Пространство $ D_1[a, b] $.} Ответ.

\que{Нормированное пространство, нормы $ \|\ \|_0 $ и $ \|\ \|_1 $ в
пространстве $ D_1[a,b] $.}

\que{Метрическое пространство, метрики $ \rho_0 $ и $ \rho_1 $ в пространстве $
D_1[a, b]$.}

\que{Метрики $ \rho_0 $ и $ \rho_1 $ в пространстве $ D_1[a, b] $. Окрестности
  точки в пространстве $ D_1[a,b] $
в смысле метрик $ \rho_0 $ и $ \rho_1 $, их связь.}

\que{Нормированное пространство. Метрическое пространство.
Связь между этими понятиями.}

\que{Слабый минимум, строгий слабый минимум, сильный минимум, строгий сильный
  минимум в классе функционалов, определённых в пространстве $ D_1[a,b] $. Связь между
слабым и сильным минимумами.}

\que{Слабый максимум, строгий слабый максимум, сильный максимум, строгий сильный
  максимум в классе функционалов, определённых в пространстве $ D_1[a,b] $. Связь между
слабым и сильным максимумами.}

\que{Непрерывный функционал в смысле близости $ k $-порядка для $ k  = 0 $ и $ k
  = 1$. Связь
между этими понятиями. Линейный функционал.}

\que{Сильный и слабый дифференциалы функционала, определённого в пространстве
$ D_1[a,b] $. Связь между этими понятиями.}

\que{Сильный и слабый дифференциалы функционала, определённого в пространстве
$ D_1[a,b] $. Необходимые условия наличия сильных и слабых экстремумов функционала.}

