\documentclass[12pt]{article}
\usepackage{amsthm, mathrsfs, mathtools, amssymb}
\usepackage[T2A]{fontenc}
\usepackage[utf8]{inputenc}
\usepackage{xcolor}

\usepackage[english,russian]{babel}
\usepackage[colorlinks = true, linkcolor = black, urlcolor = black]{hyperref}
\usepackage[depth = subsection]{bookmark}
\usepackage{enumitem}

\ifx\pdfoutput\undefined
\usepackage{graphicx}
\else
\usepackage[pdftex]{graphicx}
\fi
\usepackage{wrapfig}

 \DeclareMathOperator{\pr}{pr}
 \DeclareMathOperator{\rk}{rk}
 %\DeclareMathOperator{\arctg}{arctg}
 %\DeclareMathOperator{\arcctg}{arcctg}
 %\DeclareMathOperator{\ch}{ch}
 %\DeclareMathOperator{\sh}{sh}

\usepackage{bm}


\newtheoremstyle{example}% name
{0.7cm}% Space above
{0.7cm}% Space below
{\small}% Body font
{}% Indent amount
{\small\scshape}% Theorem head font
{.}% Punctuation after theorem head
{.5em}% Space after theorem head
{}% Theorem head spec (can be left empty, meaning ‘normal’)

\theoremstyle{example}
\newtheorem{example}{Пример}

\theoremstyle{plain}
\newtheorem{theorem}{Теорема}
\newtheorem{corollary}{Следствие}
\newtheorem*{corollary*}{Следствие} 
\newtheorem{lemma}{Лемма}
\newtheorem{utv}{Утверждение}
\newtheorem*{utv*}{Утверждение}

\theoremstyle{definition}
\newtheorem{definition}{Определение}
\newtheorem*{definition*}{Определение}
\newtheorem{question}{Вопрос}

\theoremstyle{remark}
\newtheorem{remark}{Замечание}
\newtheorem*{remark*}{Замечание}
\numberwithin{remark}{section}

\frenchspacing

\usepackage[labelsep=period]{caption}
\captionsetup{font = small}

\newcommand{\Hom}{\mathrm{Hom}}
\newcommand{\Spl}{\mathrm{Spl}}
\newcommand{\spl}{\mathrm{spl}}


\newcounter{problem} 
\newenvironment{problem}[1][]
{
	\refstepcounter{problem} 
	\par \vspace{0.7em} \noindent
  \textbf{Задача \theproblem}\ifx&#1&\else\ (#1)\fi. 
}
{
	\vspace{1em}	
}

\newenvironment{solution}
{
	\vspace{0.3em}
	\par\textsc{Решение.}
}
{
	\qed
}

\newcounter{que}
\newcommand{\que}[1]{%
  \refstepcounter{que}
	\par \vspace{0.7em} 
  \textbf{\theque.} \emph{#1}
}


\graphicspath{
    {images00-01/}{images00-02/}
      {images01-01/}{images01-02/}{images01-03/}
}
\usepackage{showlabels}
% будет показывать ссылки прямо в pdf. когда файл готов, лучше убрать

\usepackage{geometry}
\geometry{verbose,a4paper,tmargin=1cm,bmargin=2cm,lmargin=1.5cm,rmargin=1.5cm}



\begin{document}
\title{МОВИ --- тренировочный вариант и вопросы для РК №2}
\maketitle
\tableofcontents

\section*{Задачи}
\begin{problem}
Теоретический вопрос.
\end{problem}

\begin{problem}
 Найти экстремали задачи с подвижными концами.  
 \[
   \int\limits_{0}^{1} ((y')^2 + 2yy' - 32y^2)\,dx \to \mathrm{extr}.
 \]
 
\begin{solution}
  Запишем уравнение Эйлера:  
  \[
    F_y - \frac{d}{dx} F_{y'} = 0.
  \]
 Получим  
 \[
     2y' - 64y - \frac{d}{dx}(2y' + 2y) = - 64y - 2y'' = 0,
 \]
 или $ y'' + 32y = 0 $. Решим это уравнение.  
 \[
   \lambda^2 + 32 = 0 \implies \lambda = \pm\sqrt{-32} = \pm 4i\sqrt{2},\\
 \]
 Поскольку корни чисто мнимые, общее решение имеет вид  
 \begin{align*}
   y(x) &= C_1 \cos 4\sqrt{2}x + C_2 \sin 4\sqrt{2}x,\\
   y'(x) &= 4\sqrt2(-C_1\sin4\sqrt2 x + C_2\cos4\sqrt2 x).
 \end{align*}
 Запишем условия трансверсальности: $ F_{y'} \big|_{x=a} = 0 $, $ F_{y'}\big|_{x=b}=0 $,
 где $ a = 0 $, $ b = 1 $.

 В нашем случае
 \begin{multline*}
   F_{y'} = 8\sqrt{2}(-C_1\sin4\sqrt2 x + C_2 \cos4\sqrt2 x) + 2C_1\cos 4\sqrt2
   x + 2C_2 \sin4\sqrt2 x = \\ =
   (-8\sqrt2\sin4\sqrt2 x + 2\cos4\sqrt2 x)C_1 + (8\sqrt2 \cos 4\sqrt2 x +
   2\sin4\sqrt2 x) C_2.
 \end{multline*}
Получаем систему
\[
    \begin{cases}
      2C_1 + 8\sqrt2 C_2 = 0, \\
   (-8\sqrt2\sin4\sqrt2  + 2\cos4\sqrt2 )C_1 + (8\sqrt2 \cos 4\sqrt2  +
   2\sin4\sqrt2 ) C_2 = 0.
    \end{cases}
\]
Видно, что определитель $ \det A = 132 \sin4 \sqrt{2} \approx
  -77.375 \neq 0 $, поэтому система имеет единственное, нулевое решение. Вместе
  с решением нулевой будет экстремаль задачи.
\end{solution}
\end{problem}



\section*{Теоретические вопросы}
\addcontentsline{toc}{section}{Теоретические вопросы}
\vspace{-0.7em}
\que{Почему?} Потому что.





\end{document}
