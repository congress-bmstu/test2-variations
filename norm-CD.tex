Поскольку все операции (сложение и умножение на число) над функциями сохраняют
непрерывность и дифференцируемость своих аргументов (так как $ \lim[f + g] =
\lim f + \lim g $ и $ \lim \lambda f = \lambda \lim f $), то множества $ C[a,b]
$, $ D_1[a,b] $ (ровно как и $ D_k[a,b] $) являются линейными пространствами.
Стандратные нормы в вариационном исчислении --- это $ \|f\|_0 =
\max\limits_{x\in[a,b]}|f(x)| $ для
$ f(x) \in C[a,b]$ и $ \|f\|_1 = \max\limits_{x\in[a,b]}|f(x)| +
\max\limits_{x\in[a,b]}|f'(x)| $ для $ D_1[a,b] $. Таким образом, две функции $ f, g \in C[a,b] $  отличаются по норме
$ \|\ \|_0 $ менее, чем на $ \varepsilon $, если график функции $ g(x) $ целиком
лежит внутри полосы, окружающей график функции $ f(x) $ (см. рис. \ref{fig:norm_0}).
Норма $ \|\ \|_1 $ требует выполнения этого условия также для графиков $ g'(x)$,
$f'(x) $ (хотя этого и не будет достаточно).

