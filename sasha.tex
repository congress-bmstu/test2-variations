\que{Непрерывный функционал в смысле близости $ k $-порядка для $ k  = 0 $ и $ k
  = 1$. Связь
между этими понятиями. Линейный функционал.}

\paragraph{Близость $k$-го порядка.}
Рассмотрим пространство всех $k$ раз непрерывно-дифференцируемых функций $D_k$, наделённое
нормой:
\[
  \|y\|_k = \sum_{j=0}^{k} \max |y^{(j)} (x)|,
\]
где под $y^{(0)} (x)$ подразумевается сама функция. В случае $k = 0$ это пространство обозначается $C$.

Ясно, что эти пространства вложены друг в друга:
$C \supset D_1 \supset D_2 \supset \dots$, но наделены разными нормами.

\paragraph{Непрерывный функционал.}
\begin{definition}
  Функционал $\varphi(y), y \in R$ называется \emph{непрерывным в точке
  $y_0 \in R$}, если для любого $\varepsilon > 0$ существует такое $\delta > 0$,
  что $|\varphi(y) - \varphi(y_0)| < \varepsilon$ как только $\|y - y_0\| < \delta$.
\end{definition}

Функционал <<длины кривой>> $\int_a^b F(x, y, y') \, dx$, например, будет непрерывен в смысле 
нормы пространства $D_1$, но, вообще говоря, не будет таковым в смысле нормы пространства $C$,
потому что для любой кривой всегда можно указать сколь угодно близкую кривую в смысле нормы
пространства $C$ (попадающую в полоску шириной $\varepsilon$), имеющую длину во много раз
превышающую длину начальной кривой. В то же время норма пространства $D_1$ сильно ограничит
то, какие кривые мы будем называть близкими, поэтому непрерывность будет наблюдаться.

\paragraph{Линейный функционал.}
\begin{definition}
  Пусть $\varphi(h)$ -- функционал над линейным нормированным пространством $R$.
  $\varphi(h)$ называется \emph{линейным}, если 
  \begin{enumerate}
    \item $\varphi(h)$ -- непрерывный;
    \item $\forall h_1, h_2 \in R : \varphi(h_1 + h_2) = \varphi(h_1) + \varphi(h_2)$;
    \item $\forall h_1 \in R, \forall \alpha : \varphi(\alpha h) = \alpha \varphi(h)$
      \footnote{в Гельфанде-Фомине третье условие не требуется}.
  \end{enumerate}
\end{definition}


\que{Сильный и слабый дифференциалы функционала, определённого в пространстве
$ D_1[a,b] $. Связь между этими понятиями.}

Рассмотрим приращение функционала $J (y)$:
\[
  \Delta J = J(y + h) - J(y).
\]

\begin{definition}
  \emph{Сильным дифференциалом} функционала $J$ называется линейный функционал $dJ$ такой, что
  \[
    \Delta J = dJ (h) + \alpha \|h\|_0,
  \]
  где $\alpha(h) \to 0$, при $\|h\|_0 \to 0$.
\end{definition}

\begin{definition}
  \emph{Слабым дифференциалом} функционала $J$ называется линейный функционал $\delta J$ такой, что
  \[
    \Delta J = \delta J (h) + \alpha \|h\|_1,
  \]
  где $\alpha(h) \to 0$, при $\|h\|_1 \to 0$.
\end{definition}

Если $dJ$ -- сильный дифференциал, тогда $dJ$ также является и слабым дифференциалом.

\que{Сильный и слабый дифференциалы функционала, определённого в пространстве
$ D_1[a,b] $. Необходимые условия наличия сильных и слабых экстремумов функционала.}

См. предыдущий вопрос.

\begin{theorem}
  Пусть функционал $J(y)$ достигает слабого экстремума при $y = y_0$, тогда его слабый
  дифференциал (если существует) обращается в 0 при $y = y_0$.
\end{theorem}

Соответственно, для сильного экстремума равенство нулю слабого дифференциала также является необходимым условием.
