\que{Слабый минимум, строгий слабый минимум, сильный минимум, строгий сильный минимум в классе функционалов, определённых в пространстве $ D_1[a,b] $. Связь между слабым и сильным минимумами.}

\paragraph{Обозначения. } Функционал $ \mathcal{J}: K \to \mathbb{R} $, где $ K $ --- класс допустимых функций, $ K $ --- векторное пространство, $ K = C[a, b], \, D_1[a, b]$.

\begin{align*}
	&U_{0, \, \varepsilon}(y_0) = \{ y \in D_1[a, b] \mid \rho_0(y, y_0) < \varepsilon \}, \\
	&U_{1, \, \varepsilon}(y_0) = \{ y \in D_1[a, b] \mid \rho_1(y, y_0) < \varepsilon \}. \\
\end{align*}

Запись $ y \in U_{0, \, \varepsilon}(y_0) $ читается так: <<$ y $ и $ y_0 $ близки в смысле близости нулевого порядка >>. Для $  U_{1, \, \varepsilon}(y_0) $ по аналогии.

Известно, что $ U_{1, \, \varepsilon}(y_0) \subset  U_{0, \, \varepsilon}(y_0)$.

\begin{definition}
	Функционал $\mathcal{J}(y)$ достигает слабого минимума в точке $ y_0 $, если $ \exists \varepsilon > 0: \forall y \in D_1[a, b] \cap U_{1, \, \varepsilon}(y_0), \, \mathcal{J}(y) - \mathcal{J}(y_0) \geqslant 0$, причём строго слабого минимума, если $ \mathcal{J}(y) > \mathcal{J}(y_0) $ для всех $ y \not = y_0$.
\end{definition}

\begin{definition}
	Функционал $ \mathcal{J}(y) $ достигает в точке $ y_0 $ сильного минимума, если $ \exists \varepsilon > 0: \forall y \in D_1[a, b] \cap U_{0, \, \varepsilon}(y_0), \, \mathcal{J}(y) - \mathcal{J}(y_0) \geqslant 0 $, причем строго сильного минимума, если $ \mathcal{J}(y) > \mathcal{J}(y_0) $ для всех $ y \not = y_0$.
\end{definition}

\begin{utv}
	Всякий сильный экстремум является слабым экстремумом.
\end{utv}
\begin{proof}
	Пусть $ y_0 $ --- сильный экстремум, то есть $ \exists \varepsilon > 0: \forall y \in U_{0, \, \varepsilon}(y_0) \, \mathcal{J}(y) - \mathcal{J}(y_0) \geqslant~0$.
	Тогда, рассмотрим $ U_{1, \, \varepsilon}(y_0) \subset U_{0, \, \varepsilon} \Rightarrow (
	\forall y \in U_{1, \, \varepsilon}(y_0)) \, \mathcal{J}(y) - \mathcal{J}(y_0) \geqslant 0 \Rightarrow $ $ y_0 $ --- слабый экстремум.  
\end{proof}

\que{Слабый максимум, строгий слабый максимум, сильный максимум, строгий сильный максимум в классе функционалов, определённых в пространстве $ D_1[a,b] $. Связь между слабым и сильным максимумами.}

\begin{definition}
	Функционал $\mathcal{J}(y)$ достигает слабого минимума в точке $ y_0 $, если $ \exists \varepsilon > 0: \forall y \in D_1[a, b] \cap U_{1, \, \varepsilon}(y_0), \, \mathcal{J}(y) - \mathcal{J}(y_0) \leqslant 0$, причём строго слабого минимума, если $ \mathcal{J}(y) < \mathcal{J}(y_0) $ для всех $ y \not = y_0$.
\end{definition}

\begin{definition}
	Функционал $ \mathcal{J}(y) $ достигает в точке $ y_0 $ сильного минимума, если $ \exists \varepsilon > 0: \forall y \in D_1[a, b] \cap U_{0, \, \varepsilon}(y_0), \, \mathcal{J}(y) - \mathcal{J}(y_0) \leqslant 0 $, причем строго сильного минимума, если $ \mathcal{J}(y) < \mathcal{J}(y_0) $ для всех $ y \not = y_0$.
\end{definition}

\begin{utv}
	Всякий сильный экстремум является слабым экстремумом.
\end{utv}
\begin{proof}
	Пусть $ y_0 $ --- сильный экстремум, то есть $ \exists \varepsilon > 0: \forall y \in U_{0, \, \varepsilon}(y_0) \, \mathcal{J}(y) - \mathcal{J}(y_0) \leqslant~0$.
	Тогда, рассмотрим $ U_{1, \, \varepsilon}(y_0) \subset U_{0, \, \varepsilon} \Rightarrow (
	\forall y \in U_{1, \, \varepsilon}(y_0)) \, \mathcal{J}(y) - \mathcal{J}(y_0) \leqslant 0 \Rightarrow $ $ y_0 $ --- слабый экстремум.  
\end{proof}

